% SVN info for this file
%!TEX root = report.tex
\svnidlong
{$HeadURL$}
{$LastChangedDate$}
{$LastChangedRevision$}
{$LastChangedBy$}

\chapter{Conclusions and Perspectives}
\labelChapter{conclusion}

\lettrine[nindent=0pt]{I}{n} the research work leading to this thesis, an algorithm that provides an optimal, day-ahead schedule based on a proactive and reactive approach with forecast values of loads and DRES production for distribution networks is proposed. A few methods for forecasting loads and DRES are touched upon, the loads and DRES production for the two test networks used in this work for one day are then modelled, followed by the explanation of the algorithm, its various components, and its working.\\

Consequently, the algorithm was tested on the two networks, and the results clearly show that the algorithm is capable of decreasing greatly, the costs incurred by DSOs in the day-to-day operation of distribution networks. This was based on several assumptions and simplified economic models, all of which of have been earmarked for further development in future. The results obtained are very encouraging towards the further development of this idea, even though the algorithm uses simple components. The main ideas for development in future are:\\

\begin{itemize}

\item The development of a novel reconfiguration function: while the currently used function performs well, it is not state-of-the-art. However, in order to have repeatability in results, the use of genetic algorithms will be avoided unless absolutely needed.\\

\item The development of a multi-objective constrained optimization function for deciding between various methods for violation management. The current function is a simplified one, using \emph{fmincon}.\\

\item The development of an economic model that takes into account all the factors explained during the conception of simpler models. While the results shown in this thesis are interesting, one cannot apply the algorithm to a real-world situation unless such models are put in place.\\

\item The development of a day-ahead market-based purchase scenario for flexibilities. This is important from the point of view of the \emph{evolvDSO} project.

\end{itemize}
\ \\
Further developments are envisaged to be made during the three years of work that are going to be done at G2E Lab, as a part of my PhD.\\
\newpage

On a professional level, working on this thesis has enriched my knowledge in the domain of integration of renewable energies. This internship was very well suited to my interests and my career goals, which lie in finding solutions to the problems caused by the integration of renewable energies. I unequivocally believe that this will be a stepping stone for my future career in research, and that I will be able to apply what I have learnt during the course of this internship in the future.\\

On a personal level, moving to a new country in a new continent, meeting people from very different cultures during the past two years, and working with them in various capaticies has expanded my horizons, broadened my perspectives towards life, and has also been highly rewarding.