% SVN info for this file
%!TEX root = report.tex
\svnidlong
{$HeadURL$}
{$LastChangedDate$}
{$LastChangedRevision$}
{$LastChangedBy$}

\chapter{State of the Art}
\labelChapter{stateofart}

\begin{introduction}
In this chapter, the current state of research in/close to the topic of the thesis is elaborated upon. The shortcomings of current research are highlighted where applicable, and how the research leading to this thesis aims to solve them is discussed. Forecasting methods are briefly elaborated, with the aim being a more detailed presentation of the current state of research in the actual topic - distribution network optimization.
\end{introduction}



\section{Load Forecasting}

\lettrine{L}{oad} forecasting for transmission networks, done for economic dispatch, has been around for a long time now. It does help that transmission networks are almost completely observable, flexible, and built to handle problems faced by distribution networks with respect to integration of DRES. Investments to that end have been possible primarily because the cumulative length of transmission lines is less than that of distribution lines. In France, the installed length of transmission lines is about 105,000 kilometres as compared to distribution network lines, which amount to about 1.4 million kilometres. This also explains why the approach towards building transmission networks cannot be followed for distribution networks.\\

Traditionally, load forecasting has been done in transmission networks by using a multi-variable formula designed to take into account several factors. Some of them are listed below:

\begin{itemize}
\item \textbf{The season}. In France, as in several other countries present in the temperate zone of the earth, winters see consumption peaking, while during summers the consumption decreases.\\
\item \textbf{The day of the week}. Weekdays and weekends have different historic load consumption patterns.\\
\item \textbf{Holidays}. Holidays usually see a decrease in loads.\\
\item \textbf{The weather}. The forecast for the next day is always factored into the prediction method. A colder-than-normal day means more consumption while a pleasant day means lesser consumption.\\
\item \textbf{Major events}. Major sports and political events cause changes in consumption patterns and this has to be taken into account too.
\end{itemize}

Of course, this can all be done on distribution networks as well. However, the biggest problem in these networks that prevents this is their observability, rather the lack of observability in these networks. With the advent of smart meters, some of this is assuaged. Networks which see massive roll-outs of smart meters normal have better observability than their counterparts without smart meters. However, there is still a long way to go before these meters begin to be used on a large-scale.\\

Right now, one has to rely on outputs from remote terminal units (RTUs) to ascertain the various parameters in distribution networks. State estimation of distribution networks has also advanced particularly well, with researchers being able to predict the state of the network based on scarce real-time data from the networks. Research on real-time estimation of loads in distribution networks can be found from the end of the last century. In \cite{Falcao2001}, neural networks and fuzzy techniques are used to construct load curves for consumers based on their monthly consumption in kWh, and then historical load curves are used to adjust these curves to attain better accuracy. Forecast methods that don't rely on historical data are less cumbersome to design, for example, in \cite{Nose2011}, two methods for forecasting are proposed. The authors first forecast individual loads and then forecast the global load using load participation factors.

\section{Distributed Generation Forecasting}

\lettrine{F}{orecasting} of distributed generation is comparatively more difficult than forecasting of loads, owing to its intermittent, and varied nature. Different approaches have to be followed for forecasting of different types of distributed generators. The state of the art in forecasting of the production of the two major types of DRES - wind and solar, are discussed here. Generally, it is easier to forecast with only reasonable error, the production of large DRES in the short-term. As the size of DRES decreases, and as the time horizon increases, the error increases appreciably.\\

The major difference between approaches for forecasting solar and wind power is the centralized nature of parameter consideration for the latter, and the decentralized nature of parameter consideration for the former. For example, with wind power, the wind available will more or less be the same for all the generators in a wind farm. While solar irradiance will be the same too, the presence of clouds, passing objects, birds, trees, and even dust sediment can affect the production of individual solar panels. While this is not much of an issue in large solar farms that are connected at the transmission level, they affect local solar production systems on rooftops and in residential areas, and will therefore have to be taken into account in the during the formulation of the forecasting problem. This makes global forecasting for solar production extremely difficult.\\

Short-term forecasts of wind power generation continue to consider the wind direction to be constant, even today. This method, called the persistence method, also assumes that wind power output will not change. This simple method is difficult to beat in the given time horizon, sometimes even for very advanced methods, which show only around 20\% increase in accuracy, as explained in Argonne National Laboratory's report \cite{Argonne2009}, which provides a comprehensive description of the various methods used to forecast wind, the uncertainties involved, and clearly demarcates the various methods based on time horizons and evaluation methods, and elaborates on everything from numerical weather prediction methods, to handling of uncertainties in forecasts. For day-ahead forecasts, the use of numerical weather prediction methods is mandated. Some developments have been made in this domain later, with newer methods based on numerical weather prediction (\cite{Ghadi2012}) and on other methods like the one described in \cite{Bracale2013}, based on Bayesian probability.\\\ \\

For solar power forecasting, again the division can be made in terms of the time-horizon (intra-day and day-ahead). Like in the case of wind power forecasting, day-ahead forecasts will have to rely upon numerical weather prediction, and the current algorithms have about 50 km spatial resolution. In \cite{PVPS2013}, the authors explore various state-of-the-art methods for solar power forecasting, including improvement of NWP models through post-processing methods like spatio-temporal interpolation and smoothing or through combination of several NWP models. Unfortunately, the methods they describe only account for cloud shadows. Of course, as explained, in area forecasts, one cannot factor point criteria very easily. Work on this will hopefully be done in future.

\section{Distribution Network Optimization}

\lettrine{C}{lose} to forty years ago, French engineers A.\ Merlin and H.\ Back (\cite{MandB75}) devised a system to find a network configuration with reduced losses for urban distribution networks, by modifying the status of \emph{normally closed} (NC) and \emph{normally open} (NO) switches in lines. Their solution made use of a simple method that would start from a completely meshed network and progressively open lines that had the least current flowing through them, until radiality was restored. Considering the immense computational effort that would be needed for this type of an algorithm, methods based on branch exchanging among others, were also proposed (\cite{CinGra88}, \cite{ShiHon89}).\\

Computational methods today have shown that these do not provide the best solution, although in many cases, the solution is close to the best. A lot of work has been done on this topic since then, with reconfiguration methods evolving through the use of global optimization \cite{CavLyra97}. Several more advanced approaches were devised in the years that followed, ranging from heuristic methods, meta-heuristic methods, to mathematical programming. Expert system based methods have also been tried out where \emph{IF-THEN} rules are written for reacting to various conditions in the system. Predominant research has been done using meta-heuristic methods: neural networks, genetic algorithms, fuzzy theory, particle swarm optimization and other nature-inspired methods. Their advantage is that they take less time to find the optimal solution, when compared to exhaustive search methods. Other approaches use methods like mixed integer non-linear programming for reconfiguration.\\

It did not take long for researchers to realize that reconfiguration was not the only way to go in order to optimize distribution networks. Other methods, including capacitor placement, and DG allocation were tested by many researchers. Around $140$ of these research papers are tabulated and compared in \cite{KalAgn14}. A review of the literature here shows that while newer, more optimal methods have been devised for distribution networks, most of them treat the network as a static environment, i.e., only one `snapshot' of the load conditions in networks are taken into consideration for optimization. This problem is only compounded by the integration of DRES, something the available literature does not treat sufficiently. Even when DRES is considered, they are assumed to be dispatchable, which renders them next to useless in the real world.\\

As far as consideration of variability in loads is concerned for the optimization, \cite{Lopez2004} and \cite{Yin2009} use load models for one day, and solve the optimization problem using dynamic programming and binary particle swarm optimization respectively. In \cite{QueLyr2009}, the authors have used an adaptive hybrid genetic algorithm for optimizing six different networks, five of them being real-world networks. However, they have considered only loss reduction as a criterion.\\\ \\

The consideration of DRES as non-dispatchable is a certain requirement because this is what forms the core of the problem faced by DSOs today. In \cite{OlaNik2008} and \cite{SuLu2011}, the DRES are considered as non-dispatchable. Where the former fails is that it considers the DRES to produce half of their real rated power (which makes them quasi-dispatchable). The latter has only studied the effect of coordination between reconfiguration and voltage control, which can possibly increase the penetration of DRES. A more generic approach is made in \cite{Zidan2012}, where the loads for one year are modelled using the IEEE Reliability Testing System, and DRES production is modelled in different steps, with the association of a probability density function to create values for DRES production.\\

Only a few research articles have really considered creating a scheduling algorithm for distribution networks. The authors of \cite{Borghetti2010} have proposed a complete scheduling algorithm that makes use of day-ahead forecasts for renewable energy production and load consumption in distribution networks. They also add an intra-day scheduler that uses 15 minute forecasts during the day, and state estimation for unobservable, or partially observable distribution networks. The day-ahead scheduler is formulated using mixed-integer linear programming. However, the optimization is done without considering reconfiguration as one of the choices the DSO can make, and this potentially limits the options available. Load reduction is also not considered as a potential solution. Also, the optimization problem only minimizes the cost of  generation for $D-1$. In this sense, it is not complete. Therefore, there is a real need for a complete day-ahead scheduling algorithm.
