% SVN info for this file
%!TEX root = report.tex
\svnidlong
{$HeadURL$}
{$LastChangedDate$}
{$LastChangedRevision$}
{$LastChangedBy$}

\begin{abstract}

\begin{figure}[ht!]
\centering
\includegraphics[width=30mm]{Logos/evolvDSO_logo.eps}
\end{figure}
\par\vspace{5ex}

This thesis is a part of evolvDSO (Development of methodologies and tools for new and \textbf{evolv}ing \textbf{DSO} roles for efficient DRES integration in distribution networks), a three year pan-European project funded partly by the 7\ts{th} Framework Program, and launched in September 2013 with 16 partners from both academia and the industry. With the penetration of DRES and pro-active demand for electricity increasing, the power system needs to evolve in order to keep up. DSOs, the actors who are affected the most by this change have to take on new roles and responsibilities in order to continue operating distribution networks reliably and securely.\\


The evolvDSO project aims to address the main research and technology gaps that have to be solved in order to help DSOs fulfil their new responsibilities by proposing new tools and methodologies. There are eight work packages (WPs) in the project and Grenoble INP is involved in WP3 - Development of New Methods and Tools Including Simulations. The first two work packages, results of which have been used in order to perform the research leading to this thesis are \textit{Scenario writing, regulatory framework and market architectures} and \textit{Definition of use cases} respectively.\\


The major R\&D challenges in this project have been broadly clustered into four areas namely: Planning \& Network Development, Forecasting, Operational Scheduling \& Grid Optimization, Real Time Operation Maintenance, and Regulation Market Design. The scope of this work is mainly within the second area, on Grid Optimization.
\end{abstract}
