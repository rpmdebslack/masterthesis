% SVN info for this file
%!TEX root = report.tex
\svnidlong
{$HeadURL$}
{$LastChangedDate$}
{$LastChangedRevision$}
{$LastChangedBy$}

\begin{abstract}
The large-scale integration of Distributed Renewable Energy Sources (DRES) into electrical power systems, specifically distribution networks, is creating problems that have to be dealt with in future. A complete tear-down-rebuild-from-scratch approach for distribution networks might not be the best way to solve this problem. This calls for research into other possible solutions. One such solution is to counter this problem by innovatively managing the available flexibility inherent to distribution networks, which are normally open (NO) tie lines or load reduction, or reactive power injection through the converters connected to DRES. This work proposes a multi-temporal, multi-objective algorithm that makes use of forecast values of DRES production and load consumption for different distribution networks, and utilizes the aforementioned flexibilities, to provide a day-ahead optimal schedule for those networks. Some forecasting methods for loads and DRES, useful for future work, are also studied.\\
\newline
\newline
\newline
\newline
\newline
\newline
\newline

L'int\'{e}gration \`{a} grande \'{e}chelle des G\'{e}n\'{e}rateurs d'\'{E}nergie Distribu\'{e}s (GED) dans les r\'{e}seaux \'{e}lectriques, et en particulier dans les r\'{e}seaux de distribution cr\'{e}e des probl\`{e}mes qui doivent \^{e}tre trait\'{e}s \`{a} l'avenir. Une approche qui d\'{e}truit et reconstruit les r\'{e}seaux de distribution n'est pas forcement la meilleure fa\c{c}on de r\'{e}soudre ce probl\`{e}me. Cela n\'{e}cessite de rechercher d'autres solutions. Une de ces solutions est de g\'{e}rer ce probl\`{e}me en jouant d'une mani\`{e}re innovante avec la flexibilit\'{e} d\'{e}j\`{a} pr\'{e}sente dans les r\'{e}seaux de distribution, tels que des lignes normalement ouvertes (NO), la r\'{e}duction des charges, ou la puissance r\'{e}active fournit pas les convertisseurs des GED. Dans ce rapport, un algorithme multi-temporelle et multi-objectif est propos\'{e}. Il utilise les valeurs de pr\'{e}vision de la production des GED et de la consommation des charges dans diff\'{e}rents r\'{e}seaux de distribution. Il prend aussi en compte les flexibilit\'{e}s mentionn\'{e}es ci-dessus afin de fournir une planification optimale \textsc{j}-1 pour ces r\'{e}seaux. Quelques m\'{e}thodes de pr\'{e}vision des GED et des charges, utiles pour le travail \`{a} venir, sont \'{e}galement \'{e}tudi\'{e}s.
\end{abstract}
